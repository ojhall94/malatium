I had a look at your most recent version, and here are some structural comments based on that. Bear in mind that Nature Astronomy is a short-format journal, so there’s typically no available room for extensive setting of context or exhaustive literature studies… this means you really have to crystallise what it is you want to say, and just include the absolutely necessary information to support your scientific case.

 

There is too much in the way of introduction in your most recent manuscript… we don’t need a history of gyrochronology or of asteroseismology, but just the key aspects that you will build on later. For instance, you want to support the van Saders+16/19 results… that means we need to know the context of the van Saders+16 paper, the response in the literature around that time period, the impact of the van Saders+19 paper on the landscape, and then how your work will build on what has gone on previously to strengthen the evidence for weakened magnetic braking. We typically suggest that the entire introduction (so, your Section 1) should be about 500 words, and currently yours is over 1,000, so plenty of cutting can be done there. We do not usually lay out the structure of a paper (the final paragraph of S1)… Articles are so short that the structure is usually evident from the section headings.

DONE
 

Section 2 can largely be moved to the Methods section (by Methods section I mean the section that follows on from the main references in a NAstro paper, not your Section 3…), and an overall summary can be included in say, a single paragraph in the main text. Your Section 3 is very extensive, and probably should be kept together, meaning that it is probably best put into the Supplementary Information PDF.    

DONE
 

Your Section 4 can probably be kept in the main text of the paper, but you might want to reduce the level of detail… for instance, all the checks you carried out can probably be mentioned more succinctly, or given a section in the Methods.

DONE 

Discussion of your results should make up about half of the main text. Looking at your Section 5, you may want to reduce the level of detail to focus on what new information your work brings about. For instance, S5.1.2, comparison to previous studies… this looks like another checking section, so should be moved to the Methods or the SI. S5.1.3 has a lot of confirmations and comparisons… unless you are saying something new, this should be moved elsewhere. S5.2 in general, more verification… not for the main text. S5.2.5 is the key point of your paper, the main result, and this should be the highlight… this is what you set out to prove. At present it is a bit lost in the surrounding huge amounts of detail.

DONE
 

Your Conclusion section is good… in fact, it is exactly the right amount of detail for the Discussion section. So, I would say the Discussion section of your NAstro paper should be made up of your Sections 5.2.5 and 6. The actual conclusion paragraph of your NAstro paper should just be a reiteration of your main point(s), in perhaps a sentence each.

 DONE

Note that if you want software to be cited (as is good and right), you should incorporate the citations into the text somewhere (the Methods section or figure/table captions are ideal for this)… we don’t have a software section in its own right.

So, as a reminder, your main text (intro, brief outline of procedure, results, discussion, conclusion/outlook) should be about 3,000-3,500 words (not including Abstract), your Methods section should be a similar length, and everything else can go into the Supplementary Information. In terms of figures/tables: six in the main text, zero in the Methods (okay, one small table if you need it), up to 10 in an Extended Data section (caveat: all of these figures/tables should be cited from the main text/Methods) and everything else in the Supplementary Information.

 

Yes… the distinction between Methods, SI and Extended Data is a bit murky, especially as our website instructions have not caught up with our addition of Extended Data yet.



As for your questions, yes, you can move Figs. 6–8 to the Extended Data, where they’ll be renamed as ‘Extended Data Figure 1’, etc. and you can cite them from the Methods.



As for the software citations… yes, this is an issue. Unfortunately citations aren’t allowed in the Acknowledgments and citations in the Supplementary Information don’t ‘count’ (they are not indexed by Web of Science or ADS, etc.). My suggestion would be to have some citation-loaded sentences about the software packages scattered throughout the Methods… but don’t make these sound like acknowledgments or the copy editors will move them, and strip the citations… Sorry about this, it is a bit of a workaround.

