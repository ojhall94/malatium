\documentclass[12pt]{article}
\setlength{\textwidth}{167mm}
\setlength{\headsep}{0mm}
\setlength{\headheight}{0mm}
\setlength{\textheight}{230mm}
\setlength{\oddsidemargin}{1.6mm}
\setlength{\evensidemargin}{1.6mm}
\setlength{\topmargin}{4.6mm}

\usepackage{chngcntr}
\usepackage{graphicx}
\graphicspath{ {Images/} }

% Astronomical abbreviations (thanks to Dan Huber)
\newcommand{\numax}{\mbox{$\nu_{\rm max}$}\xspace}
\newcommand{\dnu}{\mbox{$\Delta \nu$}\xspace}
\newcommand{\muHz}{\mbox{$\mu$Hz}\xspace}
\newcommand{\teff}{\mbox{$T_{\rm eff}$}\xspace}
\newcommand{\logg}{\mbox{$\log(g)$}\xspace}
\newcommand{\feh}{\mbox{$\rm{[Fe/H]}$}\xspace}
\newcommand{\msun}{\mbox{$\mathrm{M}_{\odot}$}\xspace}
\newcommand{\lsun}{\mbox{$\mathrm{L}_{\odot}$}\xspace}
\newcommand{\mearth}{\mbox{$\mathrm{M}_{\oplus}$}\xspace}
\newcommand{\rsun}{\mbox{$\mathrm{R}_{\odot}$}\xspace}
\newcommand{\kepler}{\emph{Kepler}\xspace}
\newcommand{\tess}{\emph{TESS}\xspace}
\newcommand{\ktwo}{K2\xspace}
\newcommand{\gaia}{\emph{Gaia}\xspace}

\begin{document}
\noindent\textbf{\LARGE{New asteroseismic rotation rates of \emph{Kepler} dwarfs show strong agreement with weakened magnetic braking on the late-age main sequence}}\\

\noindent Oliver J. Hall$^{1,2,3}$ 
	Guy R. Davies$^{2,3}$, 
	Jennifer van Saders$^{4,5,6}$,
	Martin B. Nielsen$^{2,3}$
	Mikkel N. Lund$^{3}$, 
	William J. Chaplin$^{2,3}$, 
	Rafael A. Garc\'ia$^{7, 8}$, 
	Louis Amard$^{9}$,
	Angela A. Breimann$^{9}$, 
	Saniya Khan$^{2,3}$, 
	Victor See$^{9}$, 
	Jamie Tayar$^{4, 10}$
	\\
	
	% List of institutions
	\noindent $^{1}$ Scientific Support Office, Directorate of Science, European Space Research and Technology Centre (ESA/ESTEC), Keplerlaan 1, 2201 AZ Noordwijk, NL

	\noindent 	$^{2}$ School of Physics and Astronomy, University of Birmingham, Edgbaston, Birmingham, B15 2TT, UK

	\noindent 	$^{3}$ Stellar Astrophysics Centre, Department of Physics and Astronomy, Aarhus University, Ny Munkegade 120, 8000 Aarhus C, Denmark

	\noindent 	$^{4}$ Institute for Astronomy, University of Hawai'i, Honolulu, HI 96822

	\noindent 	$^{5}$ Observatories of the Carnegie Institution for Science, Pasadena, CA 91101

	\noindent 	$^{6}$ Department of Astrophysical Sciences, Princeton University, Princeton, NJ 08544

	\noindent 	$^{7}$ IRFU, CEA, Universit\'e Paris-Saclay, F-91191 Gif-sur-Yvette, France

	\noindent 	$^{8}$ AIM, CEA, CNRS, Universit\'e Paris-Saclay, Universit\'e Paris Diderot, Sorbonne Paris Cit\'e, F-91191 Gif-sur-Yvette, France

	\noindent 	$^{9}$ University of Exeter Department of Physics and Astronomy, Stocker Road, Devon, Exeter, EX4 4QL, UK
	
	\noindent $^{10}$ Hubble Fellow


\vspace{10mm}

\textbf{Studies using asteroseismic ages and rotation rates from spot modulation have shown that standard age-rotation relations break down roughly half-way through the main sequence lifetime, a phenomenon referred to as weakened magnetic braking. While rotation rates from spot modulation can be difficult to determine for older, less active stars, rotational splitting of asteroseismic oscillation frequencies can provide rotation rates for both active and quiescent stars \cite{adibekyan+2017}.\\
We obtained asteroseismic rotation rates of 91 main sequence stars showing high signal-to-noise modes of oscillation.
Using these new rotation rates, along with effective temperatures, metallicities and seismic masses and ages, we built a hierarchical Bayesian mixture model to determine whether the ensemble more closely agreed with a standard rotational evolution scenario, or one where weakened magnetic braking takes place. The weakened magnetic braking scenario was found to be $98.4\%$ more likely for our stellar ensemble, adding to the growing body of evidence for this stage of stellar rotational evolution. This work represents the largest catalogue of seismic rotation on the main sequence to date, opening up possibilities for more detailed ensemble analysis of rotational evolution with \textit{Kepler}.}

%%%%%%%%%%%%%%%%%%%% REFERENCES %%%%%%%%%%%%%%%%%%

% The best way to enter references is to use BibTeX:
\bibliographystyle{nature}
\bibliography{library.bib} %

\end{document}