\clearpage 
 \begin{figure*}
 	\centering
 	\includegraphics[width=.99\textwidth]{Images/modelfit.pdf}
 	\caption{A power spectrum constructed from four years of \kepler observations of 16 Cyg A (KIC 12069424). Plotted over the top is the model resulting from the fit to the data described in this work. The model implements both the mode frequencies, seen on the right hand side of the plot, and the convective background, the effects of which are seen on the left. Low frequencies have been cropped out for clarity. \textit{Inset}: A zoom in on a radial (right) and quadrupole (left) ($\ell = 0, 2$) pair of modes. The quadrupole mode is split into five components by the star's rotation. Due to the star's inclination angle with respect to us, two out of five peaks are more distinct. The height and spacing of the mode components is a function of the star's rotational splitting ($0.56\, \mu\rm{Hz}$, equivalent to $P_{\rm rot} = 20.5\, \rm{days}$) and angle of inclination ($45^\circ$).}
 	\label{fig:modelfit}
 \end{figure*}

\begin{figure}
	\centering
	\includegraphics[width=0.45\textwidth]{pgm_models.pdf}
	\caption{A probabilistic graphical model (PGM) represented algebraically in Equation \ref{eq:modelll}. The shaded circle indicates observed data, and solid black points represent other fixed information, such as the KDEs and observational uncertainties. The remaining circles represent parameters. The underline indicates that the symbol represents a set of parameters or data. Here, $\kappa_{\rm{s}}$ and $\kappa_{\rm WMB}$ represent the KDEs of standard and WMB model populations respectively. $P_{\rm{s}}$ is the mixture model weighting factor. The latent parameters $\underline{\theta}$, our observations $\underline{\mathcal{D}}$ and their uncertainties $\underline{\sigma_{\mathcal{D}}}$ include temperature (\teff), mass ($M$), log-age ($\ln(t)$), metallicity (\feh) and log-rotation ($\ln(P)$). This model is \textit{hierarchical}, as all the latent parameters are drawn from the common probability distribution set by $P_{\rm{s}}$ and described in Equation \ref{eq:mixturell}.}
	\label{fig:pgm}
\end{figure}

\begin{figure*}
	\centering
	\includegraphics[width=\textwidth]{Images/litcomp_alt2.pdf}
	\caption{Comparisons between posterior estimates of rotational parameters from this work, LEGACY and Kages \cite[private communication]{davies+2016, lund+2017}. Fractional differences are plotted against stellar rotation obtained in this work. The $\Delta$ indicates the fractional difference betwen thsi work and the literature (i.e. stars above the zero-line have higher values in this work). The right hand panels the distribution of the fractional differences around the zero line. The colour legend is consistent throughout all panels. The x-axis units on the right hand panels are equivalent to the y-axis of the left hand panels. 10 stars have been omitted from this plot and are discussed in more detail in the text: KICs 5094751, 6196457, 8349582, 8494142, 8554498, 105114430 and 11133306 all have extremely low rotation periods in Kages, with high uncertainties. Conversely, KICs 6603624, 8760414 and 8938364 have extremely high rotation periods in LEGACY with low uncertainties. In cases where stars had asymmetric error bars, the larger of the two was used when propagating uncertainty for the purposes of this figure.}
	\label{fig:legacykages}
\end{figure*}
 
\begin{figure}
 \centering
 \includegraphics[width=0.49\textwidth]{Images/seis_comparison_rot_alt2.pdf}
 \caption{Fractional differences between posterior estimates of asteroseismic rotation period from this work. Literature sources are: \cite{davies+2015} (16 Cyg A \& B), \cite{nielsen+2015} (5 stars) and \cite{benomar+2018} (40 stars). We used the reported parameter $a_{1}$ from \cite{benomar+2018}, which represents the rotational splitting in the case of uniform latitudinal rotation in their model. The dashed line represents the median of the sample shown, with the dotted lines representing the $15.9^{\rm{th}}$ and $84.1^{\rm{st}}$ percentiles. In cases where stars had asymmetric error bars, the larger of the two was used when propagating uncertainty for the purposes of this figure.}
 \label{fig:literaturecomp}
\end{figure}

 
 \begin{figure*}
 	\centering
 	\includegraphics[width=\textwidth]{Images/vsini-comparison.pdf}
 	\caption{Comparisons between asteroseismic and spectroscopic measures of projected surface rotation, $\textrm{v}\sin(i)$. All asteroseismic (x-axis) values are from this work, all spectroscopic (y-axis) values are from the literature. \textit{Left}: comparisons to 81 stars values reported in LEGACY and Kages. \textit{Right}: comparisons to 16 stars observed by \cite{benomar+2015}. Asteroseismic values are transformed from projected splitting ($\nu_s\sin(i)$) using the asteroseismic radius measurements presented in LEGACY and Kages. The solid lines indicate the 1:1 line, while the dash-dotted lines represent the 2:1 and 1:2 lines.}
 	\label{fig:vsinilit}
 \end{figure*}

%\onecolumn
\setlength\LTleft{0pt}
\setlength\LTright{0pt}
\footnotesize
\begin{longtable}{c|ccccc|ccc|ccc}
    \caption{Parameters for the 94 stars for which seismic rotation rates were obtained in this work. Temperature (\teff), age, mass, metallicity (\feh) and surface gravity ($\log(g)$) are adopted from the LEGACY \citep[L,][]{lund+2017,silvaaguirre+2017} and Kages \citep[K,][]{silvaaguirre+2015,davies+2016} catalogues, as listed in the Source column. Projected splitting ($\nu_{\rm s}\sin(i)$), inclination angle ($i$) and asteroseismic rotation ($P_{\rm rot}$) are from this work. Uncertainties were taken using the $15.9^{\rm th}$ and $84.1^{\rm st}$ percentiles of posterior distributions on the parameters, which are frequently asymmetrical in linear space. Reported values are the median of the posteriors. For parameters with no direct posterior samples (e.g. rotation) the full posterior samples were transformed before taking the summary statistics. The stellar type denotes whether a star is roughly classified as belonging to the main sequence (MS), Sub-Giants (SG) or `hot' stars (H) (see text).
    The flags indicate the following: 0; no issues, used in the gyrochronology analysis. 1; has either a number of effective samples $n_{\rm eff} < 1000$ for the asteroseismic splitting, or Gelman-Rubin convergence metric of $\hat{R} > 1.1$, indicating that rotation measurements for these stars are less robust than those with a flag of 0. 2; was found to strongly disagree with multiple literature values, excluded from the gyrochronology analysis. 3; fell outside the model range of the stellar models, and were therefore not used in the gyrochronology analysis. Table is continued on the next page.}\label{tab:results}\\
    \toprule
    KIC & $T_{\rm{eff}}$ & Age & Mass & \feh & $\log(g)$ & $\nu_{\rm{s}}\sin(i)$ & $i$   & $P_{\rm rot}$   &  Flag & Type & Source \\
        & [$\mathrm{K}$] &  [$\mathrm{Gyr}$] & [$\mathrm{M_{\odot}}$] & [$\mathrm{dex}$] & [$\mathrm{dex}$] & [$\mathrm{\mu Hz}$]  & [$\mathrm{{}^{\circ}}$] & [$\mathrm{days}$]   &  &  &  \\
    \midrule
    \endfirsthead
    \caption{\textit{Continued from previous page.}}\\
    \toprule
    KIC & $T_{\rm{eff}}$ & Age & Mass & \feh & $\log(g)$ & $\nu_{\rm{s}}\sin(i)$ & $i$   & $P_{\rm rot}$   &  Flag & Type & Source \\
        & [$\mathrm{K}$] &  [$\mathrm{Gyr}$] & [$\mathrm{M_{\odot}}$] & [$\mathrm{dex}$] & [$\mathrm{dex}$] & [$\mathrm{\mu Hz}$]  & [$\mathrm{{}^{\circ}}$] & [$\mathrm{days}$]   &  &  &  \\
    \midrule
    \endhead
    \bottomrule \multicolumn{12}{r}{\textit{Continued on next page}}\\
    \endfoot
    \bottomrule
    \endlastfoot

1435467 & 6326$\pm$77    & 3.02$_{-0.35}^{+0.50}$    & 1.32$_{-0.05}^{+0.03}$ & 0.01$\pm$0.10     & 4.100$_{-0.009}^{+0.009}$ & 1.58$_{-0.09}^{+0.10}$ & 63.4$_{-6.6}^{+10.2}$     & 6.5$_{-0.6}^{+0.8}$      & 0 &        H & L \\
2837475 & 6614$\pm$77    & 1.63$_{-0.18}^{+0.11}$    & 1.43$_{-0.02}^{+0.02}$ & 0.01$\pm$0.10     & 4.163$_{-0.007}^{+0.007}$ & 3.12$_{-0.08}^{+0.08}$ & 70.7$_{-4.4}^{+6.0}$      & 3.5$_{-0.2}^{+0.2}$      & 0 &        H & L \\
3425851 & 6343$\pm$85    & 3.32$_{-0.64}^{+0.85}$    & 1.18$_{-0.05}^{+0.05}$ & -0.04$\pm$0.10    & 4.243$_{-0.008}^{+0.008}$ & 1.17$_{-0.62}^{+0.48}$ & 60.9$_{-22.7}^{+20.1}$    & 8.1$_{-2.7}^{+8.6}$      & 0 &        H & K \\
3427720 & 6045$\pm$77    & 2.23$_{-0.24}^{+0.24}$    & 1.11$_{-0.01}^{+0.02}$ & -0.06$\pm$0.10    & 4.387$_{-0.004}^{+0.005}$ & 0.30$_{-0.06}^{+0.06}$ & 56.4$_{-23.4}^{+22.9}$    & 31.6$_{-11.8}^{+10.2}$   & 0 &        MS & L \\
3456181 & 6384$\pm$77    & 2.09$_{-0.13}^{+0.13}$    & 1.50$_{-0.02}^{+0.03}$ & -0.15$\pm$0.10    & 3.949$_{-0.009}^{+0.008}$ & 0.92$_{-0.08}^{+0.08}$ & 58.2$_{-17.7}^{+20.4}$    & 10.7$_{-2.8}^{+2.0}$     & 0 &        H & L \\
3544595 & 5669$\pm$75    & 6.63$_{-0.57}^{+0.62}$    & 0.90$_{-0.01}^{+0.01}$ & -0.18$\pm$0.10    & 4.468$_{-0.003}^{+0.003}$ & 0.40$_{-0.04}^{+0.04}$ & 66.0$_{-13.9}^{+15.6}$    & 26.1$_{-4.7}^{+3.9}$     & 0 &        MS & K \\
3632418 & 6193$\pm$77    & 2.63$_{-0.18}^{+0.18}$    & 1.41$_{-0.02}^{+0.02}$ & -0.12$\pm$0.10    & 4.024$_{-0.008}^{+0.008}$ & 0.98$_{-0.03}^{+0.03}$ & 72.3$_{-7.4}^{+10.0}$     & 11.2$_{-0.7}^{+0.6}$     & 0 &        MS & L \\
3656476 & 5668$\pm$77    & 8.37$_{-1.57}^{+1.72}$    & 1.04$_{-0.04}^{+0.05}$ & 0.25$\pm$0.10     & 4.225$_{-0.010}^{+0.008}$ & 0.21$_{-0.02}^{+0.02}$ & 62.4$_{-20.5}^{+18.9}$    & 48.0$_{-12.7}^{+8.1}$    & 0 &        MS & L \\
3735871 & 6107$\pm$77    & 2.35$_{-0.85}^{+1.04}$    & 1.09$_{-0.04}^{+0.04}$ & -0.04$\pm$0.10    & 4.396$_{-0.007}^{+0.007}$ & 0.69$_{-0.05}^{+0.05}$ & 70.4$_{-15.1}^{+13.4}$    & 15.8$_{-2.5}^{+1.8}$     & 0 &        MS & L \\
4141376 & 6134$\pm$91    & 3.27$_{-0.64}^{+0.59}$    & 1.02$_{-0.03}^{+0.02}$ & -0.24$\pm$0.10    & 4.412$_{-0.003}^{+0.004}$ & 0.76$_{-0.13}^{+0.13}$ & 64.0$_{-16.7}^{+17.5}$    & 13.4$_{-3.0}^{+3.4}$     & 0 &        MS & K \\
4143755 & 5622$\pm$106   & 11.27$_{-1.35}^{+1.50}$   & 0.92$_{-0.03}^{+0.02}$ & -0.40$\pm$0.11    & 4.102$_{-0.001}^{+0.002}$ & 0.18$_{-0.05}^{+0.08}$ & 45.8$_{-27.3}^{+30.6}$    & 48.1$_{-32.7}^{+27.4}$   & 1 &        MS & K \\
4349452 & 6270$\pm$79    & 3.45$_{-0.72}^{+0.81}$    & 1.16$_{-0.05}^{+0.04}$ & -0.04$\pm$0.10    & 4.275$_{-0.007}^{+0.008}$ & 1.50$_{-0.09}^{+0.09}$ & 79.7$_{-10.0}^{+7.1}$      & 7.5$_{-0.6}^{+0.5}$     & 0 &        H & K \\
4914423 & 5845$\pm$88    & 6.67$_{-0.62}^{+0.69}$    & 1.10$_{-0.03}^{+0.02}$ & 0.07$\pm$0.11     & 4.155$_{-0.004}^{+0.004}$ & 0.42$_{-0.14}^{+0.15}$ & 61.3$_{-30.5}^{+19.8}$    & 23.1$_{-9.6}^{+11.7}$    & 0 &        MS & K \\
4914923 & 5805$\pm$77    & 7.57$_{-1.79}^{+1.66}$    & 1.06$_{-0.05}^{+0.06}$ & 0.08$\pm$0.10     & 4.197$_{-0.010}^{+0.008}$ & 0.39$_{-0.03}^{+0.03}$ & 46.6$_{-8.1}^{+13.3}$     & 21.4$_{-3.5}^{+5.4}$     & 0 &        MS & L \\
5094751 & 5952$\pm$75    & 6.35$_{-1.05}^{+1.05}$    & 1.07$_{-0.04}^{+0.04}$ & -0.08$\pm$0.10    & 4.213$_{-0.008}^{+0.007}$ & 0.39$_{-0.16}^{+0.27}$ & 51.5$_{-31.1}^{+26.7}$    & 22.9$_{-15.8}^{+19.3}$   & 0 &        MS & K \\
5184732 & 5846$\pm$77    & 4.85$_{-0.88}^{+1.57}$    & 1.15$_{-0.06}^{+0.04}$ & 0.36$\pm$0.10     & 4.255$_{-0.008}^{+0.010}$ & 0.55$_{-0.02}^{+0.02}$ & 71.3$_{-10.8}^{+11.2}$    & 19.9$_{-1.9}^{+1.3}$     & 0 &        MS & L \\
5773345 & 6130$\pm$84    & 2.55$_{-0.24}^{+0.26}$    & 1.47$_{-0.03}^{+0.03}$ & 0.21$\pm$0.09     & 3.993$_{-0.007}^{+0.008}$ & 1.08$_{-0.08}^{+0.08}$ & 33.7$_{-2.5}^{+2.8}$      & 5.9$_{-0.5}^{+0.7}$      & 0 &        SG & L \\
5866724 & 6169$\pm$50    & 3.89$_{-0.48}^{+0.59}$    & 1.20$_{-0.03}^{+0.03}$ & 0.09$\pm$0.08     & 4.224$_{-0.005}^{+0.007}$ & 1.34$_{-0.08}^{+0.07}$ & 80.9$_{-9.4}^{+6.3}$      & 8.4$_{-0.5}^{+0.5}$      & 0 &        MS & K \\
5950854 & 5853$\pm$77    & 8.93$_{-1.15}^{+1.12}$    & 0.97$_{-0.03}^{+0.03}$ & -0.23$\pm$0.10    & 4.238$_{-0.007}^{+0.007}$ & 0.29$_{-0.12}^{+0.64}$ & 27.6$_{-15.2}^{+46.0}$    & 22.9$_{-19.7}^{+34.2}$   & 1 &        MS & L \\
6106415 & 6037$\pm$77    & 5.03$_{-1.12}^{+1.28}$    & 1.07$_{-0.04}^{+0.05}$ & -0.04$\pm$0.10    & 4.295$_{-0.009}^{+0.009}$ & 0.69$_{-0.02}^{+0.02}$ & 73.1$_{-6.3}^{+8.4}$      & 16.0$_{-0.8}^{+0.7}$     & 0 &        MS & L \\
6116048 & 6033$\pm$77    & 9.58$_{-1.90}^{+2.16}$    & 0.94$_{-0.05}^{+0.05}$ & -0.23$\pm$0.10    & 4.254$_{-0.012}^{+0.009}$ & 0.63$_{-0.02}^{+0.02}$ & 76.3$_{-10.2}^{+9.0}$     & 17.9$_{-1.2}^{+0.8}$     & 0 &        MS & L \\
6196457 & 5871$\pm$94    & 5.52$_{-0.48}^{+0.51}$    & 1.21$_{-0.03}^{+0.02}$ & 0.17$\pm$0.11     & 4.049$_{-0.004}^{+0.005}$ & 0.43$_{-0.19}^{+0.28}$ & 52.5$_{-29.1}^{+26.0}$    & 20.7$_{-13.7}^{+19.2}$   & 0 &        MS & K \\
6225718 & 6313$\pm$76    & 2.41$_{-0.43}^{+0.53}$    & 1.16$_{-0.03}^{+0.03}$ & -0.07$\pm$0.10    & 4.319$_{-0.007}^{+0.005}$ & 0.81$_{-0.03}^{+0.03}$ & 29.1$_{-1.8}^{+2.1}$      & 6.9$_{-0.5}^{+0.6}$      & 0 &        H & L \\
6278762 & 5046$\pm$74    & 11.54$_{-0.94}^{+0.99}$   & 0.74$_{-0.01}^{+0.01}$ & -0.37$\pm$0.09    & 4.560$_{-0.003}^{+0.002}$ & 0.30$_{-0.09}^{+0.09}$ & 62.2$_{-29.5}^{+19.0}$    & 33.0$_{-13.5}^{+13.6}$   & 1, 3 &        MS & K \\
6508366 & 6331$\pm$77    & 2.06$_{-0.14}^{+0.13}$    & 1.53$_{-0.02}^{+0.03}$ & -0.05$\pm$0.10    & 3.942$_{-0.007}^{+0.005}$ & 2.28$_{-0.04}^{+0.04}$ & 87.0$_{-3.2}^{+2.1}$      & 5.1$_{-0.1}^{+0.1}$      & 0 &        H & L \\
6521045 & 5825$\pm$75    & 6.50$_{-0.56}^{+0.46}$    & 1.11$_{-0.02}^{+0.02}$ & 0.02$\pm$0.10     & 4.125$_{-0.004}^{+0.004}$ & 0.45$_{-0.02}^{+0.03}$ & 75.7$_{-11.2}^{+9.7}$     & 24.8$_{-2.0}^{+1.9}$     & 0 &        MS & K \\
6603624 & 5674$\pm$77    & 7.82$_{-0.86}^{+0.94}$    & 1.01$_{-0.02}^{+0.03}$ & 0.28$\pm$0.10     & 4.320$_{-0.005}^{+0.004}$ & 1.13$_{-0.13}^{+0.13}$ & 6.9$_{-0.8}^{+0.8}$       & 1.2$_{-0.0}^{+0.0}$      & 2 &        MS & L \\
6679371 & 6479$\pm$77    & 1.95$_{-0.16}^{+0.18}$    & 1.53$_{-0.02}^{+0.04}$ & 0.01$\pm$0.10     & 3.934$_{-0.008}^{+0.007}$ & 1.90$_{-0.06}^{+0.05}$ & 82.1$_{-7.2}^{+5.5}$      & 6.0$_{-0.2}^{+0.2}$      & 0 &        H & L \\
6933899 & 5832$\pm$77    & 6.34$_{-0.62}^{+0.72}$    & 1.13$_{-0.03}^{+0.03}$ & -0.01$\pm$0.10    & 4.087$_{-0.007}^{+0.008}$ & 0.36$_{-0.02}^{+0.02}$ & 64.3$_{-14.0}^{+16.1}$    & 28.9$_{-4.8}^{+3.7}$     & 0 &        MS & L \\
7103006 & 6344$\pm$77    & 2.47$_{-0.24}^{+0.22}$    & 1.42$_{-0.02}^{+0.04}$ & 0.02$\pm$0.10     & 4.015$_{-0.007}^{+0.007}$ & 1.36$_{-0.09}^{+0.08}$ & 56.8$_{-8.8}^{+14.8}$     & 7.1$_{-1.0}^{+1.3}$      & 0 &        H & L \\
7106245 & 6068$\pm$102   & 6.27$_{-1.06}^{+1.06}$    & 0.92$_{-0.04}^{+0.02}$ & -0.99$\pm$0.19    & 4.325$_{-0.007}^{+0.007}$ & 0.32$_{-0.10}^{+0.16}$ & 33.4$_{-14.7}^{+38.4}$    & 21.4$_{-13.2}^{+23.8}$   & 1, 3 &        MS & L \\
7206837 & 6305$\pm$77    & 2.90$_{-0.30}^{+0.42}$    & 1.30$_{-0.03}^{+0.03}$ & 0.10$\pm$0.10     & 4.163$_{-0.007}^{+0.008}$ & 1.53$_{-0.12}^{+0.12}$ & 31.7$_{-2.8}^{+3.2}$      & 4.0$_{-0.4}^{+0.6}$      & 0 &        H & L \\
7296438 & 5775$\pm$77    & 7.23$_{-1.77}^{+1.49}$    & 1.08$_{-0.05}^{+0.06}$ & 0.19$\pm$0.10     & 4.201$_{-0.010}^{+0.009}$ & 0.20$_{-0.06}^{+0.06}$ & 50.5$_{-31.1}^{+28.2}$    & 45.6$_{-29.0}^{+23.4}$   & 1 &        MS & L \\
7510397 & 6171$\pm$77    & 2.82$_{-0.16}^{+0.14}$    & 1.37$_{-0.02}^{+0.02}$ & -0.21$\pm$0.10    & 4.036$_{-0.004}^{+0.007}$ & 0.64$_{-0.06}^{+0.06}$ & 19.9$_{-2.0}^{+2.0}$      & 6.1$_{-0.6}^{+0.7}$      & 0 &        MS & L \\
7670943 & 6463$\pm$110   & 2.78$_{-0.51}^{+0.62}$    & 1.24$_{-0.05}^{+0.04}$ & 0.09$\pm$0.11     & 4.228$_{-0.008}^{+0.008}$ & 1.83$_{-0.14}^{+0.14}$ & 75.7$_{-11.5}^{+9.8}$     & 6.0$_{-0.6}^{+0.6}$      & 0 &        H & K \\
7680114 & 5811$\pm$77    & 7.68$_{-1.28}^{+1.45}$    & 1.06$_{-0.05}^{+0.04}$ & 0.05$\pm$0.10     & 4.172$_{-0.010}^{+0.008}$ & 0.26$_{-0.04}^{+0.05}$ & 37.0$_{-16.3}^{+34.3}$    & 27.3$_{-13.5}^{+19.6}$   & 0 &        MS & L \\
7771282 & 6248$\pm$77    & 3.24$_{-0.32}^{+0.35}$    & 1.29$_{-0.03}^{+0.03}$ & -0.02$\pm$0.10    & 4.112$_{-0.007}^{+0.007}$ & 1.01$_{-0.18}^{+0.14}$ & 69.5$_{-17.8}^{+14.6}$    & 10.4$_{-1.7}^{+2.3}$     & 0 &        MS & L \\
7871531 & 5501$\pm$77    & 9.96$_{-1.77}^{+1.93}$    & 0.83$_{-0.02}^{+0.03}$ & -0.26$\pm$0.10    & 4.478$_{-0.005}^{+0.007}$ & 0.33$_{-0.03}^{+0.03}$ & 71.3$_{-13.2}^{+12.0}$    & 33.1$_{-4.1}^{+4.1}$     & 0 &        MS & L \\
7940546 & 6235$\pm$77    & 2.33$_{-0.08}^{+0.08}$    & 1.40$_{-0.01}^{+0.03}$ & -0.20$\pm$0.10    & 4.007$_{-0.001}^{+0.003}$ & 1.14$_{-0.03}^{+0.03}$ & 78.9$_{-7.9}^{+7.4}$      & 9.9$_{-0.4}^{+0.3}$      & 0 &        MS & L \\
7970740 & 5309$\pm$77    & 12.98$_{-2.00}^{+1.36}$   & 0.73$_{-0.01}^{+0.03}$ & -0.54$\pm$0.10    & 4.539$_{-0.005}^{+0.004}$ & 0.26$_{-0.02}^{+0.03}$ & 60.7$_{-15.3}^{+17.4}$    & 39.2$_{-9.0}^{+6.7}$     & 0 &        MS & L \\
8006161 & 5488$\pm$77    & 3.59$_{-1.45}^{+1.53}$    & 0.98$_{-0.03}^{+0.03}$ & 0.34$\pm$0.10     & 4.494$_{-0.007}^{+0.007}$ & 0.34$_{-0.02}^{+0.02}$ & 37.0$_{-3.4}^{+4.1}$      & 20.6$_{-1.8}^{+2.2}$     & 0 &        MS & L \\
8077137 & 6072$\pm$75    & 6.23$_{-1.23}^{+0.56}$    & 1.12$_{-0.05}^{+0.04}$ & -0.09$\pm$0.10    & 4.056$_{-0.013}^{+0.010}$ & 0.84$_{-0.07}^{+0.06}$ & 72.8$_{-12.6}^{+11.3}$    & 13.0$_{-1.5}^{+1.3}$     & 0 &        MS & K \\
8150065 & 6173$\pm$101   & 3.83$_{-0.67}^{+0.99}$    & 1.19$_{-0.05}^{+0.04}$ & -0.13$\pm$0.15    & 4.220$_{-0.008}^{+0.008}$ & 0.54$_{-0.13}^{+0.11}$ & 64.0$_{-21.1}^{+18.0}$    & 18.6$_{-4.9}^{+6.4}$     & 0 &        MS & L \\
8179536 & 6343$\pm$77    & 3.54$_{-0.81}^{+1.04}$    & 1.16$_{-0.06}^{+0.05}$ & -0.03$\pm$0.10    & 4.255$_{-0.010}^{+0.010}$ & 1.46$_{-0.09}^{+0.10}$ & 55.7$_{-7.4}^{+13.3}$     & 6.5$_{-0.8}^{+1.2}$      & 0 &        H & L \\
8228742 & 6122$\pm$77    & 2.89$_{-0.18}^{+0.16}$    & 1.38$_{-0.02}^{+0.02}$ & -0.08$\pm$0.10    & 4.035$_{-0.005}^{+0.007}$ & 0.64$_{-0.04}^{+0.04}$ & 37.9$_{-4.0}^{+6.2}$      & 11.0$_{-1.3}^{+2.0}$     & 0 &        MS & L \\
8292840 & 6239$\pm$94    & 3.85$_{-0.75}^{+0.81}$    & 1.15$_{-0.05}^{+0.05}$ & -0.14$\pm$0.10    & 4.240$_{-0.008}^{+0.008}$ & 1.45$_{-0.07}^{+0.07}$ & 76.2$_{-9.1}^{+8.9}$      & 7.7$_{-0.5}^{+0.5}$      & 0 &        MS & K \\
8349582 & 5699$\pm$74    & 8.03$_{-0.70}^{+0.80}$    & 1.07$_{-0.02}^{+0.02}$ & 0.30$\pm$0.10     & 4.163$_{-0.003}^{+0.004}$ & 0.23$_{-0.06}^{+0.07}$ & 59.6$_{-24.1}^{+20.7}$    & 41.7$_{-14.9}^{+19.6}$   & 1 &        MS & K \\
8379927 & 6067$\pm$120   & 1.99$_{-0.75}^{+0.85}$    & 1.12$_{-0.04}^{+0.04}$ & -0.10$\pm$0.15    & 4.388$_{-0.007}^{+0.008}$ & 1.12$_{-0.02}^{+0.02}$ & 63.3$_{-2.3}^{+2.5}$      & 9.2$_{-0.2}^{+0.3}$      & 0 &        MS & L \\
8394589 & 6143$\pm$77    & 4.45$_{-0.83}^{+0.94}$    & 1.04$_{-0.03}^{+0.04}$ & -0.29$\pm$0.10    & 4.322$_{-0.008}^{+0.008}$ & 1.01$_{-0.03}^{+0.03}$ & 71.1$_{-5.9}^{+7.9}$      & 10.9$_{-0.6}^{+0.6}$     & 0 &        MS & L \\
8424992 & 5719$\pm$77    & 9.61$_{-1.74}^{+1.92}$    & 0.92$_{-0.04}^{+0.04}$ & -0.12$\pm$0.10    & 4.359$_{-0.007}^{+0.007}$ & 0.22$_{-0.06}^{+0.06}$ & 59.1$_{-30.3}^{+21.3}$    & 42.3$_{-17.7}^{+19.4}$   & 1 &        MS & L \\
8494142 & 6144$\pm$106   & 2.62$_{-0.24}^{+0.26}$    & 1.42$_{-0.02}^{+0.03}$ & 0.13$\pm$0.10     & 4.038$_{-0.005}^{+0.005}$ & 0.67$_{-0.28}^{+0.22}$ & 62.8$_{-23.8}^{+18.5}$    & 14.5$_{-4.5}^{+9.9}$     & 0 &        MS & K \\
8554498 & 5945$\pm$60    & 5.60$_{-0.42}^{+0.45}$    & 1.20$_{-0.03}^{+0.02}$ & 0.17$\pm$0.05     & 4.007$_{-0.003}^{+0.003}$ & 0.25$_{-0.09}^{+0.21}$ & 48.1$_{-36.6}^{+30.0}$    & 35.7$_{-31.4}^{+25.6}$   & 0 &        MS & K \\
8694723 & 6246$\pm$77    & 4.69$_{-0.51}^{+0.48}$    & 1.14$_{-0.02}^{+0.02}$ & -0.42$\pm$0.10    & 4.113$_{-0.009}^{+0.007}$ & 0.92$_{-0.05}^{+0.05}$ & 34.7$_{-2.7}^{+3.4}$      & 7.2$_{-0.6}^{+0.8}$      & 0 &        MS & L \\
8760414 & 5873$\pm$77    & 11.66$_{-1.61}^{+1.28}$   & 0.81$_{-0.02}^{+0.03}$ & -0.92$\pm$0.10    & 4.329$_{-0.005}^{+0.006}$ & 0.69$_{-0.42}^{+0.26}$ & 7.6$_{-1.7}^{+2.3}$       & 2.0$_{-0.4}^{+4.5}$      & 2, 3 &        MS & L \\
8866102 & 6325$\pm$75    & 2.60$_{-0.53}^{+0.56}$    & 1.23$_{-0.04}^{+0.04}$ & 0.01$\pm$0.10     & 4.262$_{-0.007}^{+0.008}$ & 2.15$_{-0.04}^{+0.04}$ & 78.2$_{-4.7}^{+6.6}$      & 5.3$_{-0.2}^{+0.2}$      & 0 &        H & K \\
8938364 & 5677$\pm$77    & 10.25$_{-0.65}^{+0.56}$   & 0.99$_{-0.01}^{+0.01}$ & -0.13$\pm$0.10    & 4.173$_{-0.002}^{+0.007}$ & 0.61$_{-0.50}^{+0.27}$ & 8.7$_{-2.3}^{+57.5}$      & 2.0$_{-0.2}^{+87.0}$     & 2 &        MS & L \\
9025370 & 5270$\pm$180   & 6.55$_{-1.13}^{+1.26}$    & 0.97$_{-0.03}^{+0.03}$ & -0.12$\pm$0.18    & 4.423$_{-0.004}^{+0.007}$ & 0.43$_{-0.04}^{+0.04}$ & 67.5$_{-19.1}^{+15.2}$    & 24.7$_{-4.7}^{+3.7}$     & 0 &        MS & L \\
9098294 & 5852$\pm$77    & 8.08$_{-0.73}^{+0.99}$    & 0.97$_{-0.03}^{+0.02}$ & -0.18$\pm$0.10    & 4.308$_{-0.007}^{+0.005}$ & 0.36$_{-0.04}^{+0.04}$ & 58.2$_{-16.3}^{+21.0}$    & 27.2$_{-7.0}^{+5.7}$     & 0 &        MS & L \\
9139151 & 6302$\pm$77    & 1.32$_{-0.75}^{+0.94}$    & 1.18$_{-0.05}^{+0.04}$ & 0.10$\pm$0.10     & 4.382$_{-0.008}^{+0.008}$ & 0.95$_{-0.04}^{+0.04}$ & 73.5$_{-11.0}^{+11.0}$    & 11.6$_{-1.1}^{+0.8}$     & 0 &        H & L \\
9139163 & 6400$\pm$84    & 1.60$_{-0.22}^{+0.22}$    & 1.40$_{-0.02}^{+0.03}$ & 0.15$\pm$0.09     & 4.200$_{-0.008}^{+0.009}$ & 1.59$_{-0.08}^{+0.07}$ & 33.5$_{-3.0}^{+3.0}$      & 4.0$_{-0.3}^{+0.3}$      & 1 &        H & L \\
9206432 & 6538$\pm$77    & 1.53$_{-0.30}^{+0.21}$    & 1.38$_{-0.02}^{+0.04}$ & 0.16$\pm$0.10     & 4.220$_{-0.007}^{+0.005}$ & 1.55$_{-0.20}^{+0.17}$ & 34.3$_{-4.2}^{+5.7}$      & 4.1$_{-0.5}^{+1.0}$      & 0 &        H & L \\
9353712 & 6278$\pm$77    & 2.15$_{-0.13}^{+0.11}$    & 1.51$_{-0.02}^{+0.03}$ & -0.05$\pm$0.10    & 3.943$_{-0.005}^{+0.007}$ & 0.75$_{-0.17}^{+0.16}$ & 37.6$_{-12.6}^{+29.1}$    & 9.5$_{-3.9}^{+7.3}$      & 0 &        H & L \\
9410862 & 6047$\pm$77    & 6.93$_{-1.33}^{+1.49}$    & 0.97$_{-0.04}^{+0.05}$ & -0.31$\pm$0.10    & 4.300$_{-0.008}^{+0.009}$ & 0.41$_{-0.08}^{+0.09}$ & 46.4$_{-16.7}^{+28.3}$    & 20.6$_{-8.1}^{+9.8}$     & 1 &        MS & L \\
9414417 & 6253$\pm$75    & 2.65$_{-0.18}^{+0.16}$    & 1.40$_{-0.03}^{+0.02}$ & -0.13$\pm$0.10    & 4.016$_{-0.005}^{+0.005}$ & 1.09$_{-0.05}^{+0.05}$ & 58.1$_{-6.9}^{+9.7}$      & 9.0$_{-0.9}^{+1.1}$      & 0 &        H & L \\
9592705 & 6174$\pm$92    & 2.33$_{-0.16}^{+0.18}$    & 1.51$_{-0.02}^{+0.03}$ & 0.22$\pm$0.10     & 3.961$_{-0.004}^{+0.003}$ & 0.93$_{-0.09}^{+0.09}$ & 59.3$_{-12.1}^{+17.8}$    & 10.7$_{-1.9}^{+2.1}$     & 0 &        SG & K \\
9812850 & 6321$\pm$77    & 2.71$_{-0.35}^{+0.46}$    & 1.37$_{-0.05}^{+0.04}$ & -0.07$\pm$0.10    & 4.053$_{-0.009}^{+0.008}$ & 1.54$_{-0.08}^{+0.09}$ & 81.0$_{-10.0}^{+6.4}$      & 7.4$_{-0.5}^{+0.4}$     & 0 &        H & L \\
9955598 & 5457$\pm$77    & 6.29$_{-1.84}^{+1.95}$    & 0.90$_{-0.03}^{+0.04}$ & 0.05$\pm$0.10     & 4.497$_{-0.005}^{+0.007}$ & 0.29$_{-0.04}^{+0.04}$ & 53.4$_{-11.9}^{+20.8}$    & 31.4$_{-6.4}^{+9.1}$     & 0 &        MS & L \\
9965715 & 5860$\pm$180   & 2.92$_{-0.75}^{+0.86}$    & 1.21$_{-0.05}^{+0.04}$ & -0.44$\pm$0.18    & 4.272$_{-0.009}^{+0.008}$ & 1.75$_{-0.05}^{+0.05}$ & 58.3$_{-3.2}^{+3.5}$      & 5.6$_{-0.3}^{+0.3}$      & 0 &        MS & L \\
10068307 & 6132$\pm$77   & 2.36$_{-0.10}^{+0.08}$    & 1.47$_{-0.02}^{+0.01}$ & -0.23$\pm$0.10    & 3.967$_{-0.004}^{+0.004}$ & 0.71$_{-0.03}^{+0.03}$ & 41.7$_{-4.1}^{+6.0}$      & 10.9$_{-1.1}^{+1.5}$     & 0 &        SG & L \\
10079226 & 5949$\pm$77   & 3.06$_{-0.65}^{+0.70}$    & 1.12$_{-0.03}^{+0.02}$ & 0.11$\pm$0.10     & 4.366$_{-0.005}^{+0.005}$ & 0.65$_{-0.08}^{+0.07}$ & 75.1$_{-23.4}^{+10.6}$    & 16.8$_{-3.0}^{+2.2}$     & 1 &        MS & L \\
10162436 & 6146$\pm$77   & 2.46$_{-0.11}^{+0.10}$    & 1.45$_{-0.01}^{+0.02}$ & -0.16$\pm$0.10    & 3.981$_{-0.005}^{+0.005}$ & 0.84$_{-0.05}^{+0.05}$ & 25.5$_{-1.8}^{+2.1}$      & 5.9$_{-0.4}^{+0.6}$      & 0 &        SG & L \\
10454113 & 6177$\pm$77   & 2.89$_{-0.53}^{+0.56}$    & 1.17$_{-0.03}^{+0.02}$ & -0.07$\pm$0.10    & 4.314$_{-0.005}^{+0.005}$ & 0.76$_{-0.07}^{+0.07}$ & 40.9$_{-9.6}^{+26.9}$     & 10.0$_{-2.5}^{+4.8}$     & 0 &        MS & L \\
10514430 & 5784$\pm$98   & 7.84$_{-0.91}^{+0.40}$    & 1.06$_{-0.02}^{+0.04}$ & -0.11$\pm$0.11    & 4.061$_{-0.004}^{+0.004}$ & 0.18$_{-0.05}^{+0.05}$ & 57.2$_{-32.0}^{+23.5}$    & 53.6$_{-27.2}^{+23.6}$   & 1 &        MS & K \\
10516096 & 5964$\pm$77   & 7.01$_{-1.45}^{+1.33}$    & 1.06$_{-0.06}^{+0.05}$ & -0.11$\pm$0.10    & 4.169$_{-0.011}^{+0.010}$ & 0.48$_{-0.03}^{+0.03}$ & 71.8$_{-16.2}^{+12.5}$    & 22.6$_{-3.1}^{+1.9}$     & 0 &        MS & L \\
10586004 & 5770$\pm$83   & 6.43$_{-0.61}^{+0.64}$    & 1.18$_{-0.03}^{+0.02}$ & 0.29$\pm$0.10     & 4.071$_{-0.005}^{+0.005}$ & 0.48$_{-0.17}^{+0.16}$ & 59.3$_{-22.4}^{+20.6}$    & 19.6$_{-6.5}^{+11.1}$    & 1 &        MS & K \\
10644253 & 6045$\pm$77   & 2.39$_{-0.96}^{+1.12}$    & 1.10$_{-0.04}^{+0.04}$ & 0.06$\pm$0.10     & 4.396$_{-0.008}^{+0.007}$ & 0.24$_{-0.08}^{+0.08}$ & 55.6$_{-30.3}^{+24.0}$    & 38.0$_{-19.1}^{+21.1}$   & 0 &        MS & L \\
10666592 & 6350$\pm$80   & 2.11$_{-0.24}^{+0.29}$    & 1.50$_{-0.04}^{+0.04}$ & 0.26$\pm$0.08     & 4.017$_{-0.007}^{+0.009}$ & 0.91$_{-0.11}^{+0.11}$ & 47.2$_{-14.8}^{+27.2}$    & 9.5$_{-3.2}^{+3.6}$      & 0 &        H & K \\
10730618 & 6150$\pm$180  & 3.05$_{-0.29}^{+0.46}$    & 1.34$_{-0.05}^{+0.04}$ & -0.11$\pm$0.18    & 4.062$_{-0.007}^{+0.008}$ & 0.56$_{-0.25}^{+0.23}$ & 55.6$_{-26.2}^{+23.7}$    & 16.1$_{-7.2}^{+13.9}$    & 0 &        MS & L \\
10963065 & 6140$\pm$77   & 7.15$_{-1.61}^{+1.92}$    & 0.99$_{-0.06}^{+0.06}$ & -0.19$\pm$0.10    & 4.277$_{-0.011}^{+0.011}$ & 0.67$_{-0.03}^{+0.03}$ & 41.8$_{-3.6}^{+4.7}$      & 11.5$_{-1.0}^{+1.3}$     & 0 &        MS & L \\
11081729 & 6548$\pm$82   & 1.88$_{-0.42}^{+0.59}$    & 1.30$_{-0.05}^{+0.04}$ & 0.11$\pm$0.10     & 4.245$_{-0.009}^{+0.010}$ & 3.35$_{-0.10}^{+0.10}$ & 82.9$_{-5.9}^{+4.9}$      & 3.4$_{-0.1}^{+0.1}$      & 0 &        H & L \\
11133306 & 5982$\pm$82   & 5.14$_{-0.88}^{+0.86}$    & 1.06$_{-0.03}^{+0.04}$ & -0.02$\pm$0.10    & 4.314$_{-0.004}^{+0.007}$ & 0.39$_{-0.13}^{+0.14}$ & 58.0$_{-24.1}^{+21.9}$    & 24.6$_{-10.0}^{+14.3}$   & 0 &        MS & K \\
11253226 & 6642$\pm$77   & 1.60$_{-0.13}^{+0.06}$    & 1.41$_{-0.01}^{+0.02}$ & -0.08$\pm$0.10    & 4.173$_{-0.004}^{+0.005}$ & 2.55$_{-0.12}^{+0.11}$ & 49.3$_{-4.4}^{+6.3}$      & 3.4$_{-0.3}^{+0.4}$      & 0 &        H & L \\
11295426 & 5793$\pm$74   & 6.31$_{-0.34}^{+0.32}$    & 1.07$_{-0.02}^{+0.01}$ & 0.12$\pm$0.07     & 4.280$_{-0.003}^{+0.003}$ & 0.22$_{-0.03}^{+0.03}$ & 55.6$_{-20.9}^{+22.9}$    & 42.6$_{-14.5}^{+11.5}$   & 0 &        MS & K \\
11401755 & 5911$\pm$66   & 7.10$_{-0.59}^{+0.61}$    & 1.06$_{-0.02}^{+0.03}$ & -0.20$\pm$0.06    & 4.039$_{-0.004}^{+0.004}$ & 0.55$_{-0.10}^{+0.09}$ & 64.6$_{-19.5}^{+17.4}$    & 18.5$_{-4.3}^{+4.7}$     & 0 &        MS & K \\
11772920 & 5180$\pm$180  & 10.67$_{-2.97}^{+2.73}$   & 0.83$_{-0.04}^{+0.04}$ & -0.09$\pm$0.18    & 4.500$_{-0.008}^{+0.005}$ & 0.31$_{-0.04}^{+0.03}$ & 70.8$_{-14.3}^{+12.6}$    & 35.1$_{-4.7}^{+5.3}$     & 1 &        MS & L \\
11807274 & 6225$\pm$75   & 3.59$_{-0.45}^{+0.78}$    & 1.24$_{-0.04}^{+0.04}$ & 0.00$\pm$0.08     & 4.135$_{-0.007}^{+0.009}$ & 1.42$_{-0.07}^{+0.07}$ & 76.8$_{-9.4}^{+8.8}$      & 7.9$_{-0.5}^{+0.5}$      & 0 &        MS & K \\
11853905 & 5781$\pm$76   & 6.71$_{-0.67}^{+0.77}$    & 1.12$_{-0.03}^{+0.02}$ & 0.09$\pm$0.10     & 4.102$_{-0.005}^{+0.004}$ & 0.27$_{-0.09}^{+0.09}$ & 54.7$_{-28.8}^{+24.5}$    & 34.0$_{-17.2}^{+18.9}$   & 1 &        MS & K \\
11904151 & 5647$\pm$74   & 10.23$_{-0.67}^{+0.83}$   & 0.92$_{-0.02}^{+0.01}$ & -0.15$\pm$0.10    & 4.344$_{-0.003}^{+0.003}$ & 0.24$_{-0.07}^{+0.06}$ & 64.9$_{-28.4}^{+18.1}$    & 40.9$_{-12.8}^{+16.5}$   & 1 &        MS & K \\
12009504 & 6179$\pm$77   & 3.97$_{-0.43}^{+0.57}$    & 1.17$_{-0.04}^{+0.02}$ & -0.08$\pm$0.10    & 4.211$_{-0.005}^{+0.007}$ & 1.16$_{-0.04}^{+0.03}$ & 71.3$_{-5.7}^{+8.0}$      & 9.4$_{-0.5}^{+0.5}$      & 0 &        MS & L \\
12069127 & 6276$\pm$77   & 2.01$_{-0.13}^{+0.11}$    & 1.57$_{-0.02}^{+0.03}$ & 0.08$\pm$0.10     & 3.912$_{-0.004}^{+0.005}$ & 0.54$_{-0.33}^{+1.26}$ & 16.6$_{-6.3}^{+54.2}$     & 5.2$_{-3.9}^{+39.9}$     & 1 &        H & L \\
12069424 & 5825$\pm$50   & 6.67$_{-0.77}^{+0.81}$    & 1.05$_{-0.02}^{+0.02}$ & 0.10$\pm$0.03     & 4.287$_{-0.007}^{+0.007}$ & 0.40$_{-0.01}^{+0.01}$ & 44.7$_{-2.9}^{+6.2}$      & 20.5$_{-1.1}^{+2.0}$     & 1 &        MS & L \\
12069449 & 5750$\pm$50   & 7.39$_{-0.91}^{+0.89}$    & 0.99$_{-0.02}^{+0.02}$ & 0.05$\pm$0.02     & 4.353$_{-0.005}^{+0.007}$ & 0.31$_{-0.01}^{+0.01}$ & 34.0$_{-2.4}^{+3.0}$      & 21.2$_{-1.5}^{+1.8}$     & 0 &        MS & L \\
12258514 & 5964$\pm$77   & 4.05$_{-0.16}^{+0.18}$    & 1.26$_{-0.01}^{+0.01}$ & 0.00$\pm$0.10     & 4.126$_{-0.003}^{+0.004}$ & 0.39$_{-0.03}^{+0.04}$ & 35.0$_{-6.6}^{+22.9}$     & 16.7$_{-3.6}^{+10.0}$    & 1 &        MS & L \\
12317678 & 6580$\pm$77   & 2.46$_{-0.18}^{+0.22}$    & 1.34$_{-0.01}^{+0.04}$ & -0.28$\pm$0.10    & 4.048$_{-0.009}^{+0.008}$ & 1.27$_{-0.14}^{+0.13}$ & 35.3$_{-5.2}^{+10.1}$     & 5.2$_{-0.9}^{+1.8}$      & 0 &        H & L \\
\end{longtable}
\normalsize
\twocolumn

 